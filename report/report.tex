\documentclass[a4paper,12pt]{article}
\usepackage[utf8x]{inputenc}
\usepackage[LGR]{fontenc}
\usepackage{ucs}

\usepackage{polyglossia}
\setdefaultlanguage{greek}

\usepackage{listings}
\usepackage{textcomp}

\setmainfont{Times New Roman}
\setsansfont{Arial}
\newfontfamily\greekfont[Script=Greek]{Times New Roman}

\DeclareGraphicsExtensions{.pdf, .jpg}


%----------------------------------------------------------------------------------------
%	LISTINGS (CODE) TEMPLATE
%----------------------------------------------------------------------------------------

\lstset
{
	keywordstyle=\bfseries\ttfamily\color[rgb]{0,0,1},
	identifierstyle=\ttfamily,
	commentstyle=\color[rgb]{0.133,0.545,0.133},
	stringstyle=\ttfamily\color[rgb]{0.627,0.126,0.941},
	showstringspaces=false,
	basicstyle=\small,
	numberstyle=\footnotesize,
	numbers=left,
	stepnumber=1,
	numbersep=10pt,
	tabsize=2,
	breaklines=true,
	prebreak = \raisebox{0ex}[0ex][0ex]{\ensuremath{\hookleftarrow}},
	breakatwhitespace=false,
	aboveskip={1.5\baselineskip},
  	columns=fixed,
  	upquote=true,
  	extendedchars=true,
	frame=single
	inputencoding=utf8
}

\begin{document}

\begin{titlepage}

\newcommand{\HRule}{\rule{\linewidth}{0.5mm}} 

\center
 
%----------------------------------------------------------------------------------------
%	HEADING SECTION
%----------------------------------------------------------------------------------------

\textsc{\LARGE Εθνικό Μετσόβιο Πολυτεχνείο}\\[1.5cm] % Name of your university/college
\textsc{\Large Βάσεις Δεδομένων}\\[0.5cm] % Major heading such as course name


%----------------------------------------------------------------------------------------
%	TITLE SECTION
%----------------------------------------------------------------------------------------

\HRule \\[0.4cm]
{ \huge \bfseries Αναφορά Εξαμηνιαίο Project 2013-2014  }\\[0.4cm]
\HRule \\[1.5cm]
 
%----------------------------------------------------------------------------------------
%	LOGO SECTION
%----------------------------------------------------------------------------------------

\includegraphics[scale=0.5]{ntua_logo} 
 
%----------------------------------------------------------------------------------------
%	AUTHOR SECTION
%----------------------------------------------------------------------------------------
\vfill

Ηλίας Φωτόπουλος \\ 03109106\\
Θανάσης Βράτιμος \\03110ΧΧΧ


%----------------------------------------------------------------------------------------

\end{titlepage}

\section{Λεπτομέρειες Υλοποίησης}
Για την κατασκευή του project έγινε χρήση των παρακάτω τεχνολογιών:
\begin{itemize}
  \item Ruby on Rails (Framework)
  \item HTML \ Sass (Syntactically Awesome Style Sheets) 
  \item Bootstrap (Sass Framework)
  \item CoffeeScript
  \item Git (Version Control)
\end{itemize}

Κατά την φάση του σχεδιασμού αποφασίστηκε η χρήση του RoR framework λόγω της αρχιτεκτονικής MVC που ακολουθεί αλλά και λόγω της φιλοσοφίας του Convention over Configuration (CoC) και Don't Repeat Yourself (DRY).




\end{document}
