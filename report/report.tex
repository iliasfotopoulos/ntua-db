\documentclass[a4paper,12pt]{article}
\usepackage[utf8x]{inputenc}
\usepackage[LGR]{fontenc}
\usepackage{ucs}

\usepackage{polyglossia}
%\setdefaultlanguage{greek}

\usepackage{listings}
\usepackage{textcomp}

\setmainfont{Times New Roman}
\setsansfont{Arial}
\newfontfamily\greekfont{Times New Roman}
\setmainfont[Script=Greek]{Times New Roman}



\DeclareGraphicsExtensions{.pdf, .jpg}


%----------------------------------------------------------------------------------------
%	LISTINGS (CODE) TEMPLATE
%----------------------------------------------------------------------------------------

\lstset
{
	keywordstyle=\bfseries\ttfamily\color[rgb]{0,0,1},
	identifierstyle=\ttfamily,
	commentstyle=\color[rgb]{0.133,0.545,0.133},
	stringstyle=\ttfamily\color[rgb]{0.627,0.126,0.941},
	showstringspaces=false,
	basicstyle=\small,
	numberstyle=\footnotesize,
	numbers=left,
	stepnumber=1,
	numbersep=10pt,
	tabsize=2,
	breaklines=true,
	prebreak = \raisebox{0ex}[0ex][0ex]{\ensuremath{\hookleftarrow}},
	breakatwhitespace=false,
	aboveskip={1.5\baselineskip},
  	columns=fixed,
  	upquote=true,
  	extendedchars=true,
	frame=single
	inputencoding=utf8
}

\begin{document}

\begin{titlepage}

\newcommand{\HRule}{\rule{\linewidth}{0.5mm}} 

\center
 
%----------------------------------------------------------------------------------------
%	HEADING SECTION
%----------------------------------------------------------------------------------------

\textsc{\LARGE Εθνικό Μετσόβιο Πολυτεχνείο}\\[1.5cm] % Name of your university/college
\textsc{\Large Βάσεις Δεδομένων}\\[0.5cm] % Major heading such as course name


%----------------------------------------------------------------------------------------
%	TITLE SECTION
%----------------------------------------------------------------------------------------

\HRule \\[0.4cm]
{ \huge \bfseries Αναφορά Εξαμηνιαίο Project 2013-2014  }\\[0.4cm]
\HRule \\[1.5cm]
 
%----------------------------------------------------------------------------------------
%	LOGO SECTION
%----------------------------------------------------------------------------------------

\includegraphics[scale=0.5]{ntua_logo} 
 
%----------------------------------------------------------------------------------------
%	AUTHOR SECTION
%----------------------------------------------------------------------------------------
\vfill

Ηλίας Φωτόπουλος \\ 03109106\\
Θανάσης Βράτιμος \\03110769


%----------------------------------------------------------------------------------------

\end{titlepage}

\section{Λεπτομέρειες Υλοποίησης}
Για την κατασκευή του project έγινε χρήση των παρακάτω τεχνολογιών:
\begin{itemize}
  \item Ruby on Rails (Framework)
  \item HTML \ Sass (Syntactically Awesome Style Sheets) 
  \item Bootstrap (Sass Framework)
  \item CoffeeScript
  \item Git (Version Control)
\end{itemize}

Κατά την φάση του σχεδιασμού αποφασίστηκε η χρήση του RoR framework λόγω της αρχιτεκτονικής MVC που ακολουθεί αλλά και λόγω της φιλοσοφίας του Convention over Configuration (CoC) και Don't Repeat Yourself (DRY).

\section{Constraints}
Τα constraints και η business logic της εφαρμογής μας γίνεται εξολοκλήρου μέσα στο framework της RoR (εκτός από τα 2 triggers, βλ.Section Triggers).
Σε κάθε φόρμα για update ή edit μιας σχέσης γίνονται κάποια validations.\\
\subsection{Hotel Validations}
	\begin{lstlisting}[language=Ruby]
		validates :name, :city, presence: true
		validates :street_number, numericality: true
		validates :rating, numericality: true, 
		inclusion: {in: 0..5, message: "rating should be 0-5"}
	
		validates :construction_year, :renovation_year,
		inclusion: { in: 1800..Date.today.year, message: "Year should be over 1800"},
	  	allow_nil: true,
	  	format: 
	  	{ 
	    	with: /(18|19|20)\d{2}/i, 
	    	message: "should be a four-digit year"
	  	}
	\end{lstlisting}

\subsection{Room Validations}
	\begin{lstlisting}[language=Ruby]
	validates :price, :room_type, presence: true
	validates :price, numericality: {greater_than: 0}
	\end{lstlisting}
	
\subsection{Client Validations}
	\begin{lstlisting}[language=Ruby]
	validates :identity, :first_name, :last_name, presence: true
	validates :identity, uniqueness: {message: "This identity already exists. Identity must be unique!"}
	validates :street_number, numericality: true
	\end{lstlisting}
	
\subsection{Reservation Validations}
	Ειδικά για τις Reservation φτιάξαμε μια Service με όνομα ReservationService, η οποία ουσιαστικά ελέγχει να δει αν ένα δωμάτιο είναι ελεύθερο κάποιες ημερομηνίες που της δίνουμε. Η βασική της λογική φαίνεται στην παρακάτω συνάρτηση:
	
	\begin{lstlisting}[language=Ruby]
	  def room_available?
	    reservations = UpcomingReservationUpdatable.where("room_id = ?", @room.id)
	    arrival_date = Date.parse @arrival
	    departure_date = Date.parse @departure
	
	    reservations.each do |r|
	      before = (arrival_date < r.arrival_date) && ( departure_date < r.departure_date)
	      after = (arrival_date > r.arrival_date) && ( departure_date > r.departure_date)
	      if before || after
	        next
	      else
	        return false
	      end
	    end
	    return true
	  end
	\end{lstlisting}
	Και τα γενικότερα validation του model Reservation:
	\begin{lstlisting}[language=Ruby]
		belongs_to :client
		validates :client_id, presence: true
	
		belongs_to :room
		validates :room_id, presence: true
	
		has_one :hotel, through: :room
		validates :hotel_id, presence: true
	
		validates :arrival_date, :departure_date, presence: true
	
		validate :arrival_before_departure, :arrival_date_cannot_be_in_the_past, :departure_date_cannot_be_in_the_past
	
		def arrival_before_departure
		    errors.add(:arrival_date, "must be before departure date") if arrival_date >= departure_date
	  	end
	
	  	def arrival_date_cannot_be_in_the_past
		    errors.add(:arrival_date, "can't be in the past") if arrival_date < Date.today
	  	end
	
	  	def departure_date_cannot_be_in_the_past
		    errors.add(:departure_date, "can't be in the past") if departure_date < Date.today
	  	end
	\end{lstlisting}

\section{SQL Queries}
Σε όλες τις σχέσεις δημιουργήθηκαν queries για προβολή, εισαγωγή, ενημέρωση και διαγραφή εγγραφών, τα οποία δεν περιλαμβάνονται την παρούσα αναφορά αφού δεν χρειάζονται κάποια εξήγηση.

	\subsection{Join queries}
	Γίνεται συχνή χρήση ερωτημάτων Join στην εφαρμογή μας όπως για παράδειγμα στο Reservations index:
	\begin{lstlisting}[language=SQL]
	SELECT reservations.id, reservations.arrival_date, reservations.departure_date, reservations.created_at, reservations.updated_at, clients.first_name, clients.last_name, hotels.name, rooms.room_type
	FROM reservations
	INNER JOIN clients
	ON reservations.client_id = clients.id
	INNER JOIN hotels
	ON reservations.hotel_id = hotels.id
	INNER JOIN rooms
	ON reservations.room_id = rooms.id
	ORDER BY arrival_date
	\end{lstlisting}
	Άλλο ένα παράδειγμα χρήσης Join είναι στο show client, όπου εμφανίζονται και οι reservations του κάθε πελάτη:
	\begin{lstlisting}[language=SQL]
	SELECT reservations.id, reservations.arrival_date, reservations.departure_date, reservations.created_at, reservations.updated_at, hotels.name, rooms.room_type
	FROM reservations
	INNER JOIN hotels
	ON reservations.hotel_id = hotels.id
	INNER JOIN rooms
	ON reservations.room_id = rooms.id
	ORDER BY arrival_date
	\end{lstlisting}
	Ενώ μπορείτε να δείτε και άλλο παράδειγμα χρήσης join στα Views που δημιουργήσαμε για την εφαρμογή.
	
	\subsection{Aggregate queries}
	Στην εφαρμογή μας κάναμε κυρίως χρήση του count για να μετράμε τον αριθμό των reservations ενός πελάτη ή τον αριθμό των δωματίων ενός ξενοδοχείου.\\

	Εύρεση αριθμού δωματίων ενός ξενοδοχείου:  
	\begin{lstlisting}[language=SQL]
	SELECT COUNT(*) 
	FROM rooms 
	WHERE rooms.hotel_id = hotel_id
	\end{lstlisting}
	
	Εύρεση αριθμού κρατήσεων ενός πελάτη: 
	\begin{lstlisting}[language=SQL]
	SELECT COUNT(*) 
	FROM reservations 
	WHERE reservations.client_id = client_id
	\end{lstlisting}
	
	\subsection{Group By}
	Στην ενότητα Reports της εφαρμογής γίνεται χρήση της εντολής Group By σε συνδυασμό με την aggregate εντολή count.
	
	Το παρακάτω ερώτημα βρίσκει τον αριθμό των κρατήσεων κάθε ξενοδοχειακής μονάδας.
	\begin{lstlisting}[language=SQL]
	SELECT COUNT(*) as count, hotels.name as name
	FROM reservations
	INNER JOIN hotels
	ON reservations.hotel_id = hotels.id
	GROUP BY hotels.name
	\end{lstlisting}
	Ενώ το παρακάτω βρίσκει τον αριθμό των κρατήσεων ανά τύπο δωματίου.
	\begin{lstlisting}[language=SQL]
	SELECT COUNT(*) as count, rooms.room_type as room_type
	FROM reservations
	INNER JOIN rooms
	ON reservations.room_id = rooms.id
	GROUP BY rooms.room_type
	\end{lstlisting}
	
	\subsection{Order By}
	Έχει γίνει χρήση Order By στα περισσότερα ερωτήματα της εφαρμογής μας. Μερικά παραδείγματα υπάρχουν στα views που εμφανίζονται στο παρακάτω section της παρούσας αναφοράς.
	
	\subsection{Group By - Having}
	Το παρακάτω ερώτημα επιστρέφει τα ξενοδοχεία που έχουν αριθμό δωματίων μεγαλύτερου του 5.
	\begin{lstlisting}[language=SQL]
	SELECT COUNT(*) as count, hotels.name as name
	FROM rooms
	INNER JOIN hotels
	ON rooms.hotel_id = hotels.id
	GROUP BY hotels.id
	HAVING count > 5
	\end{lstlisting}

\section{Views}
Δημιουργήσαμε τα παρακάτω δύο views.
Το view upcoming reservations περιέχει όλες τις κρατήσεις για τις οποίες η μέρα άφιξης είναι μεταγενέστερη της σημερινής μέρας. Επίσης εκτελεί inner join, με τα tables: clients,hotels,rooms έτσι ώστε να εμφανίσει όνομα πελάτη, ξενοδοχείου αλλά και τον τύπο του δωματίου. Η χρησιμότητα του view είναι η παρουσίαση όλων των κρατήσεων για τις οποίες ο πελάτης δεν έχει φτάσει ακόμη.

\begin{lstlisting}[language=SQL]
CREATE OR REPLACE VIEW upcoming_reservations AS
SELECT reservations.id, reservations.arrival_date, reservations.departure_date, reservations.created_at, reservations.updated_at, clients.first_name, clients.last_name, hotels.name, rooms.room_type
FROM reservations
INNER JOIN clients
ON reservations.client_id = clients.id
INNER JOIN hotels
ON reservations.hotel_id = hotels.id
INNER JOIN rooms
ON reservations.room_id = rooms.id
WHERE arrival_date > CURDATE()
ORDER BY arrival_date
\end{lstlisting}

Το view upcoming reservations updatable επιτελεί τις ίδιες λειτουργίες με το παραπάνω με δύο βασικές διαφορές:
\begin{itemize}
  \item Δεν χρησιμοποιεί inner join για να πάρει τα names από τα άλλα tables.
  \item Περιέχει κρατήσεις όπου η μέρα αναχώρησης (όχι άφιξης) είναι μεταγενέστερη της σημερινής
\end{itemize}
Το εν λόγω view παίζει σημαντικό ρόλο στην εσωτερική αρχιτεκτονική και λογική της εφαρμογής μας. Χρησιμοποιείται από το ReservationService το οποίο ελέγχει αν ένα δωμάτιο είναι διαθέσιμο για κράτηση.

\begin{lstlisting}[language=SQL]
CREATE OR REPLACE VIEW upcoming_reservations_updatable AS
SELECT *
FROM reservations
WHERE departure_date > CURDATE()
ORDER BY arrival_date
\end{lstlisting}

\section{Triggers}
Δημιουργήσαμε τα παρακάτω δύο triggers.\\
To trigger delete hotel rooms, το οποίο πριν διαγράψει ένα ξενοδοχείο διαγράφει και όλα τα δωμάτια του. Τα δωμάτια ανήκουν στο  αδύναμο σύνολο οντοτήτων "Δωμάτιο" και εξαρτιούνται πλήρως από το ξενοδοχείο στο οποίο ανήκουν, οπότε και δεν έχει νόημα να υφίστανται αν διαγραφεί το ξενοδοχείο τους.

\begin{lstlisting}[language=SQL]
CREATE TRIGGER delete_hotel_rooms
BEFORE DELETE ON hotels 
FOR EACH ROW
BEGIN
	DELETE FROM rooms WHERE OLD.id = hotel_id;
END
\end{lstlisting}

Παρομοίως με παραπάνω δημιουργήσαμε ένα trigger που διαγράφει όλες τις κρατήσεις ενός πελάτη πριν διαγραφεί ο πελάτης.

\begin{lstlisting}[language=SQL]
CREATE TRIGGER delete_client_reservations
BEFORE DELETE ON clients 
FOR EACH ROW
BEGIN
	DELETE FROM reservations WHERE OLD.id = client_id;
END
\end{lstlisting}

Στο σημείο αυτό αξίζει να αναφερθεί ότι το παραπάνω θα μπορούσε να υλοποιηθεί εύκολα και μέσα από το περιβάλλον του RoR Framework (χρήση Rails ActiveRecord callbacks). Αρκεί να προστεθούν τα παρακάτω στα μοντέλα hotel και client αντίστοιχα.

\begin{lstlisting}[language=Ruby]
has_many :rooms, dependent: :destroy
\end{lstlisting}

\begin{lstlisting}[language=Ruby]
has_many :reservations, dependent: :destroy
\end{lstlisting}

\subsection{Callback vs Trigger}
	\textbf{Πλεονεκτήματα Callback:}
	\begin{itemize}
	  \item Όλη η Business Logic του μοντέλου βρίσκεται στην Rails κάτι το οποίο κάνει πιο εύκολη την συντήρηση και ανάπτυξη του.
	  \item Εύκολο Debug
	  \item Ο κώδικας Ruby γράφετε και διαβάζεται πιο εύκολα
	  \item Η εφαρμογή μας δεν εξαρτάται από την υλοποίηση (και συντακτικό) της βάσης δεδομένων.
	\end{itemize}
	\textbf{Πλεονεκτήματα Trigger:}
	\begin{itemize}
	  \item Είναι πιο γρήγορο από ένα callback. (Στο callback χρειάζεται να συνδεθείς στην βάση δεδομένων, ενώ στο trigger είσαι ήδη στο layer της βάσης δεδομένων)
	\end{itemize}
	Βάση των παραπάνω και δεδομένου το μέγεθος της εφαρμογής το ιδανικό θα ήταν να γίνει χρήση Rails ActiveRecord callbacks.


\end{document}
